\documentclass[twocolumn]{article}

\providecommand{\keywords}[1]{\noindent \textbf{\textit{Keywords}} #1}

\begin{document}

\title{Eventually Consistent Partying}
\author{Veit Heller}

\maketitle

\begin{abstract}
In distributed systems and life in general, eventual consistency is the
desirable state of agreement. In this paper, we show how to reach this state in
the context of parties with regards to the buzz factor (also known as
inebriation quotient).
\end{abstract}

\bigskip

\keywords{Party System Design, Buzz Factor}

\section{Introduction}

In classical distributed systems, eventual consistency is a state of agreement
between nodes in a system that is reached at a certain point in time. This
property is usually desirable because it provides clear grounds on which to
base assumptions about the state of the system.

At parties, too, there is a certain state of agreement—or agreeability, if you
want—that is usually beneficial to the mood of the actors in the system, a
property worth optimizing for. This property is in strong correlation with the
buzz factor—also known as inebriation quotient—, which describes the state of
alcohol saturation of a given actor.

In this paper, we describe a novel approach to reason about eventual
consistency as it relates to partying.

\section{Preliminaries}

Our foremost goal for eventual consistency in a party setting is reaching a
quorum. It is said to be almost impossible to set up a perfectly consistent
system unless you set up a group of exclusively sober actors. There is no
consensus on whether a group of sober actors can be classified as a party in
the literature, as some authors prefer to call those systems “tea parties”.

We characterize an actor by their buzz intake behaviour and their buzz
threshold. If the buzz threshold is reached, we characterize an actor as having
dropped out and disregard them when checking for a quorum. The buzz threshold
seems to be correlated to the actor’s gender, age, and bodyweight.

\subsection{Archetypes}

In our research we discovered four primary archetypes of actors at parties in
relation to attaining the buzz factor. We will describe them in the following.

\paragraph{Serious Business.} Actors in the “Serious Business” class—also known
as “winos” or “boozehounds”—are characterized by a quick intake of buzz.
Depending on their buzz threshold, they might drop out early.

\paragraph{Dead Sober.} Actors in the “Dead Sober” class—also known as
“buzzkills” or “designated drivers”—do not intake buzz. Accordingly, they either
have to be in the majority to achieve what is known as an \textit{a priori}
quorum, or be in the minority, in which case they will work against the quorum
of the other actors.

\paragraph{Low and Slow.} The “Low and Slow” class of actors—also known as
“normies”—shows a slow, steady intake of buzz. Any spikes in buzz may be
counteracted by the intake of water or other non-alcoholic substances, though
the increase of buzz over time is almost inevitable.

\paragraph{Sugar Rush.} Favoring Cocktails and longdrinks, this class of
actors—also known as “fancypants” or “amateur baristas”—show almost no signs of
buzz for a period of time—this period is unpredictable and largely based on the
compounds used when intaking buzz—before a sudden spike of buzz that can easily
overshoot the buzz threshold.

\bigskip

While this collection of archetypes is helpful in building a vocabulary for
expressing actor behavior, it should not be taken as canonical or exhaustive.
Party architecture is an understudied area of systems design, and we expect many
novel, more accurate categorizations to emerge in the coming years.

\section{Practical Discussion}

Now that we have laid the groundwork to conceptualize the system, we can talk


\section{Conclusion}

\end{document}
