\documentclass[9pt]{sigplanconf}
\usepackage[utf8]{inputenc}

\title{Eventually Consistent Partying}
\authorinfo{Veit Heller}{}{}

\begin{document}

\maketitle

\begin{abstract}
In distributed systems and life in general, eventual consistency is the
desirable state of agreement. In this paper, we show how to reach this state in
the context of parties with regards to the buzz factor (also known as
inebriation quotient).
\end{abstract}

\section{Introduction}

In classical distributed systems, eventual consistency is a state of agreement
between nodes in a system that is reached at a certain point in time. This
property is usually desirable because it provides clear grounds on which to
base assumptions about the state of the system.

At parties, too, there is a certain state of agreement—or agreeability, if you
want—that is usually beneficial to the mood of the actors in the system, a
property worth optimizing for. This property is in strong correlation with the
buzz factor—also known as inebriation quotient—, which describes the state of
alcohol saturation of a given actor.

In this paper, we describe a novel approach to reason about eventual
consistency as it relates to partying.

\section{Terminology}

- buzz factor related to archetype & bodyweight (citation)

- archetype
  - heavyweight: lots of alcohol
  - dead sober: no alcohol
  - low and slow: a little
  - sugar rush: cocktails & longdrinks

\end{document}
